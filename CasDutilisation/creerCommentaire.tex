\documentclass[a4paper,10pt]{article}
\usepackage[francais]{babel}
\usepackage[utf8]{inputenc}
\usepackage[dvips]{geometry}

\title{Cas D'utilisation : admin}
\author{}
%opening
\begin{document}

\maketitle



\textbf{Domaine:} journal en ligne\\
\textbf{Titre:} créer un commentaire\\
\textbf{Résumé:} Liste des actions réalisées par un abonné souhaitant créer un article.\\
\textbf{Acteurs:} Abonné(A)\\
\textbf{Motivation:} A souhaite créer un commentaire.\\
\textbf{Pré-conditions:} L'utilisateur A s'est loggué en tant qu'abonné et est sur la page de consultation
d'un article.\\
\textbf{post-condition(s):} Le commentaire est crée par le systeme.\\

\textbf{Remarques ergonomiques}
\textbf{Contraintes non fonctionnelles}
\begin{center}
  \begin{tabular}{|c|c|}
\hline
   \textbf{Type de contrainte}&\textbf{Descriptif}\\
\hline
Temps de réponse&De l'ordre de la seconde\\
\hline
Fréquence & Au maximum 1 secondes\\
\hline
Disponibilité & Cette fonction doit être opérationnelle à tout instant dans le système\\
\hline
Concurrence&\\
\hline
Intégrité&Non spécifique.\\
\hline
Confidentialité& Non spécifique.\\
\hline

  \end{tabular}
\end{center}

\textbf{Action de départ}

\textbf{Scénario nominal}
\begin{enumerate}
\item A active la fonctionnalité su systeme qui crée un commentaire.
\item Le systeme affiche la page permettant de créer un article. 
\item R saisit le texte pour créer son commentaire.
\item Le systeme confirme que le commentaire a bien été crée et redirige sur la page de consultation de l'article.
\end{enumerate}

\textbf{Action de fin:}

10 R quitte la fonctionnalité du système qui crée un commentaire.
\end{document}