\documentclass[a4paper,10pt]{article}
\usepackage[francais]{babel}
\usepackage[utf8]{inputenc}
\usepackage[dvips]{geometry}

\title{Cas D'utilisation : admin}
\author{}
%opening
\begin{document}

\maketitle



\textbf{Domaine:} journal en ligne\\
\textbf{Titre:} créer un article\\
\textbf{Résumé:} Liste des actions réalisées par un redacteur souhaitant créer un article.\\
\textbf{Acteurs:} Redacteur(R)\\
\textbf{Motivation:} R souhaite créer un article.\\
\textbf{Pré-conditions:} L'utilisateur R s'est loggué en tant que redacteur.\\
\textbf{post-condition(s):} L'article est crée par le systeme.\\

\textbf{Remarques ergonomiques}
\textbf{Contraintes non fonctionnelles}
\begin{center}
  \begin{tabular}{|c|c|}
\hline
   \textbf{Type de contrainte}&\textbf{Descriptif}\\
\hline
Temps de réponse&De l'ordre de la seconde\\
\hline
Fréquence & Au maximum 1 secondes\\
\hline
Disponibilité & Cette fonction doit être opérationnelle à tout instant dans le système\\
\hline
Concurrence&\\
\hline
Intégrité&Non spécifique.\\
\hline
Confidentialité& Non spécifique.\\
\hline

  \end{tabular}
\end{center}

\textbf{Action de départ}

\textbf{Scénario nominal}
\begin{enumerate}
\item R active la fonctionnalité du systeme qui crée un article.
\item R saisit les informations qui permettent de créer un article(texte et titre).
\item Le systeme confirme que l'article a bien été crée.
\end{enumerate}

\textbf{Action de fin:}

10 R quitte la fonctionnalité du système qui crée un article.
\end{document}