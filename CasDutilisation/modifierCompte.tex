\documentclass[a4paper,10pt]{article}
\usepackage[francais]{babel}
\usepackage[utf8]{inputenc}
\usepackage[dvips]{geometry}

\title{Cas D'utilisation : admin}
\author{}
%opening
\begin{document}

\maketitle



\textbf{Domaine:} journal en ligne\\
\textbf{Titre:} modifier un compte\\
\textbf{Résumé:} Liste des actions réalisées par un admin souhaitant modifier un compte.\\
\textbf{Acteurs:} admin(A)\\
\textbf{Motivation:} L'admin souhaite modifier un compte.\\
\textbf{Pré-conditions:} Le compte existe, L'utilisateur admin s'est loggué en tan qu'admin.\\
\textbf{post-condition(s):} Le compte est modifié et validé par le systeme.\\

%\textbf{Exception:}\\ Exception A : L'Organisateur ne valide pas la liste

%Le systeme revient à l'état suivant l'action de départ.

\textbf{Remarques ergonomiques}
\textbf{Contraintes non fonctionnelles}
\begin{center}
  \begin{tabular}{|c|c|}
\hline
   \textbf{Type de contrainte}&\textbf{Descriptif}\\
\hline
Temps de réponse&De l'ordre de la seconde\\
\hline
Fréquence & Au maximum 10 secondes\\
\hline
Disponibilité & Cette fonction doit être opérationnelle à tout instant dans le système\\
\hline
Concurrence&Non applicable, il n'y a qu'un admin et un seul.\\
\hline
Intégrité&Non spécifique.\\
\hline
Confidentialité& Non spécifique.\\
\hline

  \end{tabular}
\end{center}

\textbf{Action de départ}

\textbf{Scénario nominal}
\begin{enumerate}
\item A active la fonctionnalité du systeme qui modifie un compte.
\item A saisit les informations qui permettent d'identifier un compte.
\item Le systeme affiche la page qui permet de modifier ce compte.
\item A modifie le compte.
\item Le systeme confirme que le compte a bien été modifié.
\end{enumerate}

\textbf{Action de fin:}

10 A quitte la fonctionnalité du système qui modifie un compte.
\end{document}
