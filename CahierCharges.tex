\documentclass[a4paper,11pt]{article}
\usepackage[francais]{babel}
\usepackage[utf8]{inputenc}
\usepackage{amsmath}
\usepackage{graphicx}
\usepackage{amssymb}
\usepackage{geometry}
\usepackage{listings}
\usepackage{color}

\title{Cahier des charges - Groupe 2}
\author{Pierre BIENAIMÉ, Bastien BONNET, Mathieu CHATAIGNER,\\ Mathieu FRESQUET, Paul GAY}
%\geometry{hmargin=.2\textwidth, vmargin=.15\textheight}

\begin{document}
\maketitle



\section{Description du site}
Il s'agit d'un site de journal gratuit en ligne. L'accès aux articles sera possible en fonction du niveau d'autorisation de l'utilisateur courant. 

\section{Liste des fonctionnalités du produit final}

\subsection{Fonctions disponibles}
    \begin{description}
	\item[Rédaction d'articles en ligne :] Possibilité, pour les utilisateurs disposant d'un com\-pte avec niveau d'autorisation approprié ( \texttt{Rédacteur} ), de rédiger et de publier un article en ligne ;
	\item[Consultation d'articles :] Possibilité, pour les utilisateurs disposant d'un compte avec niveau d'autorisation approprié ( \texttt{Abonné} ), de consulter des articles en ligne ;
	\item[Commentaires d'articles :] Possibilité, pour les utilisateurs disposant d'un compte avec niveau d'autorisation approprié ( \texttt{Abonné} ), de poster un commentaire sur un article. Ce commentaire sera visible par tous les gens disposant des droits de lecture de l'article concerné.
    \end{description}

\subsection{Création / gestion de comptes}
    \begin{description}
	\item[Niveaux de confidentialité :] Quatre niveaux de confidentialité seront définis :
	           \begin{itemize}
		    \item \texttt{Invité} : il peut consulter le dernier article publié et voir les résultats d'une recherche (la liste des titres d'articles envoyés en réponse à la recherche), sans pour autant pouvoir consulter les articles complets ;
		    \item \texttt{Abonné} : il dispose des droits de lecture pour tous les articles, de recherche d'articles ainsi que du droit de poster un commentaire d'article. Il peut de plus modifier ses propres commentaires ;
		    \item \texttt{Rédacteur} : il dispose des droits de l'              \texttt{Abonné} ainsi que du droit de rédaction d'article. Il peut également modifier ses propres articles ;
		    \item \texttt{Administrateur} : Il dispose des droits du 	             \texttt{Rédacteur} ainsi que du droit de modification de tous les articles et commentaires d'articles présents sur le site. Il peut également supprimer des comptes d'utilisateurs de type \texttt{Abonné}.
		    \end{itemize}

	\item[Création / Edition de compte :] Possibilité pour les personnes ayant un profil        \texttt{In\-vité} de créer un compte pour obtenir un statut d'        \texttt{Abonné}.
    \end{description}

\subsection{Recherche d'articles}
    Possibilité d'effectuer une recherche d'articles avec les critères suivants :
    \begin{itemize}
	\item Par mots clefs ;
	\item Par auteur ;
	\item Par Date.
    \end{itemize}

\subsection{Valorisation des comptes}
    \begin{description}
	\item[Système de vote :] Chaque abonné aura la possibilité de voter une fois pour chaque article pour donner son avis :
	\begin{itemize}
	 \item Bon
	\item Moyen
	\item Mauvais
	\end{itemize}

    Pour un même auteur, une moyenne des avis sera effectuée afin d'attribuer une note à chaque auteur.

	\item [Points en fonction du nombre d'articles consultés :] Chaque abonné gagnera un point par nouvel article consulté (donc un article déjà consulté aupravant ne comptera pas), ce qui reflètera le degré d'intérêt qu'il porte au site.
    \end{description}


% \definecolor{light-gray}{gray}{0.95}
% \lstset{language=octave,
% frame=shadowbox,
% backgroundcolor=\color{light-gray},
% basicstyle=\footnotesize,
% stringstyle=\ttfamily,
% xleftmargin=1cm,
% xrightmargin=1cm,
% breaklines,
% showstringspaces=false,
% showtabs=false,
% extendedchars=false,
% numbers=left,
% numberstyle=\tiny,
% stepnumber=5,
% numbersep=5pt}

% \begin{lstlisting}
% coucou
% \end{lstlisting}


\end{document}